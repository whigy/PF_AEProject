\documentclass[12pt]{article}
\usepackage[utf8]{inputenc}
\usepackage{nameref}
\usepackage{subcaption}
\usepackage{changepage}
\usepackage{graphicx}
\usepackage[usenames, dvipsnames]{color}
\usepackage[a4paper,includeheadfoot,margin=3cm]{geometry}
\usepackage[nottoc,numbib]{tocbibind}
\newcommand{\bigcell}[2]{\begin{tabular}{@{}#1@{}}#2\end{tabular}}
%\usepackage{titlesec}
%\setcounter{secnumdepth}{4}
\usepackage[table]{xcolor}

\usepackage{floatrow}
\DeclareFloatFont{tiny}{\scriptsize}% "scriptsize" is defined by floatrow, "tiny" not
\floatsetup[table]{font=scriptsize}

%\titleformat{\paragraph}
%{\normalfont\normalsize\bfseries}{\theparagraph}{1em}{}
%\titlespacing*{\paragraph}
%{0pt}{3.25ex plus 1ex minus .2ex}{1.5ex plus .2ex}

\bibliographystyle{vancouver}


\title{Particle Filter based Object Detection\\
	{\Large EL2320 Applied Estimation Project}}
\author{Huijie Wang}
\date{January 2016}

\begin{document}

\maketitle

\begin{abstract}
	\par
	In this project, a particle filter is implemented to track selected object in the video. Two methods of observation models are tested: single colour comparison and histogram based colour distribution model. As a result, the latter method to a large extend reduces the risk of lost tracking according to colour variance, as it is a more reliable way to represent real world object. However, it also shows to be sensitive to colour especially when illumination changes or background noises.
\end{abstract}

\section{Introduction}
\par
Object tracking is a typical computer vision problem, with regard to the problem of 

\section{Background}

\section{Methods}

\subsection{Particle Filter}

\subsection{Motion Model}
As as 

\subsection{Observation Model}
\subsubsection{Single Colour model}

\subsubsection{Colour distribution model}

\subsection{Resampling}

\section{Result}

\section{Discussion}
% Motion model

% Color: should pay more attention on red

	

\bibliography{references}
	
\end{document}